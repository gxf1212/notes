% !TeX root = demo.tex

\chapter{Theoretical of foundations of classical statistical mechanics}

\section{Overview}

It's statistical mechanics' job to communicate between macroscopic thermodynamics and microscopic laws of motion. Thermodynamics is a phenomenological theory which makes no reference to the microscopic constituents of matter; however, we can rationalize thermodynamics based on microscopic mechanical laws illustrated in chapter 1. But problems emerge:
\begin{itemize}
	\item macroscopic systems possess an enormous number of degree of freedom, or number of particles;
	\item in real-world systems, the interactions between particles are highly nontrivial;
	\item (\textit{Loschmidt's paradox}) microscopic mechanical laws are "inherently reversible", while the second law of thermodynamics prescribes a direction of time.
\end{itemize}
People then realized that macroscopic properties of a system do not depend strongly on the motion of every particle, but rather on gross averages that "wash out" microscopic details. 

The principal conceptual breakthrough is that of an \textit{ensemble}, which refers to a collection of systems that share common macroscopic properties. This chapter is mainly about this genius idea.

\section{The laws of thermodynamics}

Newton's second law:
\begin{equation}
	\vF=m\dd{\vr}{t}=m\ddot{\vr} 
\end{equation}

\section{Overview}



\section{Overview}


\section*{Summary of chapter}

\begin{itemize}
	\item 
	\item 
\end{itemize}